\section {Notes}

\begin{enumerate}[label=(\roman*)]
\item Section 21.1 describes the operations supported by a disjoint-set data
  structure.
\item Section 21.2 describes the linked-list implementation for disjoint sets.
\item Section 21.3 presents a more efficient representation using rooted trees.
\item Section 21.4 analysis of union by rank with path compression.
\end{enumerate}

\section {Disjoint-set operations}

\textbf{MAKE-SET}($x$) creates a new set whose only member is $x$. Since the
sets are disjoint, we require that $x$ not already be in some other set.

\textbf{UNION}($x, y$) unites the dynamic sets that contain $x$ and $y$, say
$S_x$ and $S_y$, into a new set that is the union of these two sets.

\textbf{FIND-SET}($x$) returns a pointer to the representative of the (unique)
set containing $x$.

\subsection {An application of disjoint-set data structure}

The procedure CONNECTED-COMPONENTS that follows uses the disjoint-set operations
to compute the connected components of a graph. Once CONNECTED-COMPONENTS has
preprocessed the graph, the procedure SAME-COMPONENT answers queries about
whether two vertices are in the same connected component.

\textbf{CONNECTED-COMPONENTS($G$)}\\
1\hspace*{2ex} \textbf{for} each vertex $v \in G.V$\\
2\hspace*{4ex} MAKE-SET($v$)\\
3\hspace*{2ex} \textbf{for} each edge $(u, v) \in G.E$\\
4\hspace*{4ex} \textbf{if} FIND-SET$(u) \neq$ FIND-SET($v$)\\
5\hspace*{6ex} UNION($u, v$)

\textbf{SAME-COMPONENT($u, v$)}\\
1\hspace*{2ex} \textbf{if} FIND-SET($u$) $==$ FIND-SET($v$)\\
2\hspace*{4ex} \textbf{return} TRUE\\
3\hspace*{2ex} \textbf{else return} FALSE

\subsection {Exercises}

\subsubsection {21.1-1}

Suppose that CONNECTED-COMPONENTS is run on the undirected graph $G = (V, E)$,
where $V = { a, b, c, d, e, f, g, h, i, j, k }$ and the edges of $E$ are
processed in the order $(d, i), (f, k), (g, i), (b, g), (a, h), (i, j), (d, k),$
$(b, j), (d, f), (g, j), (a, e)$. List the vertices in each connected component
after each iteration of lines3-5.

\textbf{Solution}

\begin{tabular}{l|l}
Edge processed & Collection of disjoint sets \\
\hline
initial sets   & \{a\} \{b\} \{c\} \{d\} \{e\} \{f\} \{g\} \{h\} \{i\} \{j\} \{k\} \\
(d, i)         & \{a\} \{b\} \{c\} \{d, i\} \{e\} \{f\} \{g\} \{h\} \{j\} \{k\} \\
(f, k)         & \{a\} \{b\} \{c\} \{d, i\} \{e\} \{f, k\} \{g\} \{h\} \{j\} \\
(g, i)         & \{a\} \{b\} \{c\} \{d, i, g\} \{e\} \{f, k\} \{h\} \{j\} \\
(b, g)         & \{a\} \{b, d, g, i\} \{c\} \{e\} \{f, k\} \{h\} \{j\} \\
(a, h)         & \{a, h\} \{b, d, g, i\} \{c\} \{e\} \{f, k\} \{j\} \\
(i, j)         & \{a, h\} \{b, d, g, i, j\} \{c\} \{e\} \{f, k\} \{j\} \\
(d, k)         & \{a, h\} \{b, d, f, g, i, j, k\} \{c\} \{e\} \{j\} \\
(b, j)         & \{a, h\} \{b, d, f, g, i, j, k\} \{c\} \{e\} \{j\} \\
(d, f)         & \{a, h\} \{b, d, f, g, i, j, k\} \{c\} \{e\} \{j\} \\
(g, j)         & \{a, h\} \{b, d, f, g, i, j, k\} \{c\} \{e\} \{j\} \\
(a, e)         & \{a, e, h\} \{b, d, f, g, i, j, k\} \{c\} \{j\} \\
\end{tabular}

\subsubsection {21.1-2}



