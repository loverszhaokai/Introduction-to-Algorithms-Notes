\section {Notes}

\begin{enumerate}[label=(\roman*)]
\item There maybe more than one MST in a forest.
\item The number of all the edges in the MST is equal to $\boldsymbol{V} - 1$.
\end{enumerate}

\section {Growing a minimum spanning tree}

\subsection {Definition}

\subsubsection {A}
A is a subset of some minimum spanning tree.  

\subsubsection {Safe edge}
Safe edge is a edge that add to A and A is also a subset of some minimum
spanning tree.

\subsection {Generic-MST}

\textbf{GENERIC-MST}(G, $w$)\\
1\space $A$ = $\varnothing$\\
2\space \textbf{while} $A$ does not form a spanning tree\\
3\space\space\space\space find an edge $(u, v)$ that is
\textbf{safe edge} for $A$\\
4\space\space\space\space $A = A \cup \{(u, v)\}$\\
5\space return $A$\\
\\
\textbf{Initialization:} After line 1, the set A trivially satisfies the loop
invariant.\\
\textbf{Maintenance:} The loop in lines 2-4 maintains the invariant by adding
only safe edges.\\
\textbf{Termination:} All edges added to A are in a minimum spanning tree, and
so the set A returned in line 5 must be a minimum spanning tree.

\subsection {Theorem}

Let $G=(V, E)$ be a connected, undirected graph with a real-valued weight
function $\omega$ defined on $E$. Let $A$ be a subset of $E$ that is inclued in
some minimum spanning tree for $G$, let $(S, V - S)$ be any cut of G that
respects A, and let $(u, v)$ be a light edge crossing $(S, V - S)$. Then, edge
$(u, v)$ is \textbf{safe} for A. {\color{red} Namely, $ A \cup {(u, v)}$ is also
included in some minimum spanning tree for $G$.}

\textbf{Proof} Let $T$ be a minimum spanning tree that includes $A$, and
{\color{red}assume that $T$ does not contain the light edge $(u, v)$}, since if
it does, the edge is obviously \textbf{safe} for A. We shall construct another
minimum spanning tree $T^{'}$ that includes $A \cup {(u, v)}$ by using
cut-and-paste technique, thereby showing that $(u, v)$ is a \textbf{safe} edge
for $A$.

The edge $(u, v)$ forms a {\color{red}cycle} with the edges on the simple path
$p$ from $u$ to $v$ in $T$. Since $u$ and $v$ are on opposite sides of the cut
$(S, V - S)$, at least one edge in $T$ lies on the simple path $p$ and also
crosses the cut. Let $(x, y)$ be any such edge. The edge $(x, y)$ is not in $A$,
because the cut respects $A$. Since $(x, y)$ is on the unique simple path from
$u$ to $v$ in $T$, removing $(x, y)$ breaks $T$ into two components. Adding
$(u, v)$ reconnects them to form a new spanning tree
$T^{'} = T - \{(x, y)\} \cup \{(u, v)\}$.

We next show that $T^{'}$ is a minimum spanning tree. Since $(u, v)$ is a light
edge crossing $(S, V - S)$ and $(x, y)$ also crosses this cut,
$w(u, v) \leq  w(x, y)$. Therefore,
$w(T^{'}) = w(T) - w(x, y) + w(u, v) \leq w(T)$.

When $w(T^{'}) == w(T)$, we know that $T^{'}$ is also a minimum spanning tree,
so the edge $(u, v)$ is \textbf{safe} for $A$.

When $w(T^{'}) < w(T)$, since we let $T$ be a minimum spanning tree and
\textbf{assume} that $T$ does not contain the light edge $(u, v)$. Therefore,
the \textbf{assume} is false, so $T$ must contain the light edge $(u, v)$, and
the edge $(u, v)$ is \textbf{safe} for $A$.

\subsection {Corollary}

TODO: Prim -> Kruskal
TODO: Kruskal -> Prim

