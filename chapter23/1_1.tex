\section {Notes}

\begin{enumerate}[label=(\roman*)]
\item There maybe more than one MST in a forest.
\item The number of all the edges in the MST is equal to $\boldsymbol{V} - 1$.
\end{enumerate}

\section {Growing a minimum spanning tree}

\subsection {Definition}

\subsubsection {A}
A is a subset of some minimum spanning tree.  

\subsubsection {Safe edge}
Safe edge is a edge that add to A and A is also a subset of some minimum
spanning tree.

\subsection {Generic-MST}

\textbf{GENERIC-MST}(G, $w$)\\
1\space $A$ = $\varnothing$\\
2\space \textbf{while} $A$ does not form a spanning tree\\
3\space\space\space\space find an edge $(u, v)$ that is
\textbf{safe edge} for $A$\\
4\space\space\space\space $A = A \cup \{(u, v)\}$\\
5\space return $A$\\
\\
\textbf{Initialization:} After line 1, the set A trivially satisfies the loop
invariant.\\
\textbf{Maintenance:} The loop in lines 2-4 maintains the invariant by adding
only safe edges.\\
\textbf{Termination:} All edges added to A are in a minimum spanning tree, and
so the set A returned in line 5 must be a minimum spanning tree.

\subsection {Theorem 1.}

Let $G=(V, E)$ be a connected, undirected graph with a real-valued weight
function $\omega$ defined on $E$. Let $A$ be a subset of $E$ that is inclued in
some minimum spanning tree for $G$, let $(S, V - S)$ be any cut of G that
respects A, and let $(u, v)$ be a light edge crossing $(S, V - S)$. Then, edge
$(u, v)$ is \textbf{safe} for A. {\color{red} Namely, $ A \cup {(u, v)}$ is also
included in some minimum spanning tree for $G$.}

\textbf{Proof} Let $T$ be a minimum spanning tree that includes $A$, and
{\color{red}assume that $T$ does not contain the light edge $(u, v)$}, since if
it does, the edge is obviously \textbf{safe} for A. We shall construct another
minimum spanning tree $T^{'}$ that includes $A \cup {(u, v)}$ by using
cut-and-paste technique, thereby showing that $(u, v)$ is a \textbf{safe} edge
for $A$.

The edge $(u, v)$ forms a {\color{red}cycle} with the edges on the simple path
$p$ from $u$ to $v$ in $T$. Since $u$ and $v$ are on opposite sides of the cut
$(S, V - S)$, at least one edge in $T$ lies on the simple path $p$ and also
crosses the cut. Let $(x, y)$ be any such edge. The edge $(x, y)$ is not in $A$,
because the cut respects $A$. Since $(x, y)$ is on the unique simple path from
$u$ to $v$ in $T$, removing $(x, y)$ breaks $T$ into two components. Adding
$(u, v)$ reconnects them to form a new spanning tree
$T^{'} = T - \{(x, y)\} \cup \{(u, v)\}$.

We next show that $T^{'}$ is a minimum spanning tree. Since $(u, v)$ is a light
edge crossing $(S, V - S)$ and $(x, y)$ also crosses this cut,
$w(u, v) \leq  w(x, y)$. Therefore,
$w(T^{'}) = w(T) - w(x, y) + w(u, v) \leq w(T)$.

When $w(T^{'}) == w(T)$, we know that $T^{'}$ is also a minimum spanning tree,
so the edge $(u, v)$ is \textbf{safe} for $A$.

When $w(T^{'}) < w(T)$, since we let $T$ be a minimum spanning tree and
\textbf{assume} that $T$ does not contain the light edge $(u, v)$. Therefore,
the \textbf{assume} is false, so $T$ must contain the light edge $(u, v)$, and
the edge $(u, v)$ is \textbf{safe} for $A$.

\subsection{Exercises}

\subsubsection {23.1-1}

Let $(u, v)$ be a minimum-weight edge in a connected graph $G$. Show that
$(u, v)$ belongs to some minimum spanning tree of $G$.

\textbf{Solution}

Let $E_u$ be all the edges that connected to the point $u$.

a. If there is only one edge connected to the point $u$, the edge belongs to
\textbf{all} the minimum spanning tree of $G$.

b. If there is more than one edge connected to the point $u$, we assume that
{\color{red} $(u, v)$ is not in any minimum spanning trees of $G$}. There must
be one edge $(u, x) { x != v}$ that is in some minimum spanning tree of $G$,
since $w(u, v) < w(u, x)$, therefore, the edge $(u, x)$ can not be in some
minimum spanning tree of $G$. So there is conflict and the assume is false. So,
the $(u, v)$ belongs to some minimum spanning tree of $G$.

\subsubsection {23.1-2}

Professor Sabatier conjectures the following converse of Theorem 1. in Minimum
Spanning Tree. Let $G = (V, E)$ be a connected, undirected graph with a
real-valued weight function $w$ defined on $E$. Let $A$ be a subset of $E$ that
is included in some minimum spanning tree for $G$, let $(S, V - S)$ be any cut
of $G$ that respects $A$,and let $(u, v)$ be a safe edge for $A$ crossing
$(S, V - S)$. Then, $(u, v)$ is a light edge for the cut. Show that the
professor’s conjecture is incorrect by giving a counterexample.

\textbf{Solution}

a. Here is a special case, the point $v$ of $(u, v)$ only has one edge, and
$w(u, v)$ is the largest, let $(x, y)$ be any other edge that crosses the cut,
obviously, $(u, v)$ is not a light edge for the cut.

b. Here is a generic case, assume that there is a light edge $(u^{'}, v^{'})$
crossing the cut and the edge has no common point with $(u, v)$, so
$w(u^{'}, v^{'}) < w(u, v)$. After combine $A$ with $(u^{'}, v^{'})$, there is
another $cut^{'}$ that crossing $(u, v)$, and it is a light edge for $cut^{'}$.
The previous case shows that $(u, v)$ is not a light edge for any cut but some
cut when $(u, v)$ is a safe edge for $A$.

\subsubsection {23.1-3}

Show that if an edge $(u, v)$ is contained in some minimum spanning tree, then
it is a light edge crossing some cut of the graph.

\textbf{Solution}

Let $T$ be the minimum spanning tree that contains the edge $(u, v)$, if we
remove the edge from $T$, and the other edges are $A$, obviously there is some
cut that crosses the edge $(u, v)$ which respects $A$. Then we are going to show
that the edge $(u, v)$ is a light edge crossing these cut.

If there is only one edge crossing the cut, obviously the edge $(u, v)$ is a
light edge crossing the cut.

If there is more than one edge crossing the cut, let $(x, y)$ be any edges
crossing the cut other than $(u, v)$. Assume that $w(x, y) < w(u, v)$, there
will another minimum spanning tree $T^{'}$ and
$w(T^{'}) = w(T) - \{(u, v)\} + \{(x, y)\} < w(T)$ which is impossible since the
$T$ is a minimum spanning tree. So the assume is contradiction and
$w(x, y) \geq w(u, v)$, so the edge $(u, v)$ is a light edge crossing some cut
of the graph.

\subsubsection {23.1-4}

Give a simple example of a connected graph such that the set of edges
\{$(u, v)$: there exists a cut $(S, V - S)$ such that $(u, v)$ is a light edge
crossing $(S, V - S)$\} does not form a minimum spanning tree.

\textbf{Solution}

There is a quadrangle: $V={A,B,C,D}, E={(A,B), (A,C), (B,C), (B,D), (C,D)}$
$, w(A,B)=w(A,C)=w(B,C)=1, w(B,D)=w(C,D)=2$. Obviously, $(A,B), (A,C) and (B,C)$
are lights edges crossing some cut. So the tree edges can join the set. And they
construct a circle, so the set can not form a minimum spanning tree.

I think if we add {\color{red} respect} to the set, then the set will form a
minimum spanning tee. Such as, \{$(u, v)$: there exists a cut $(S, V - S)$
{\color{red} which repects this set} such that $(u, v)$ is a light edge
crossing $(S, V - S)$\}, and the set will form a minimum spanning tree.

\subsubsection {23.1-5}

Let $e$ be a maximum-weight edge on some cycle of connected graph $G = (V, E)$.
Prove that there is a minimum spanning tree of $G^{'} = (V, E - {e})$ that is
also a minimum spanning tree of $G$. That is, there is a minimum spanning tree
of $G$ that does not include $e$.

\textbf{Solution}

Assume that there is a minimum spanning tree $T$ of $G$ including the edge $e$.
Firstly, we construct a tree $T^{'}$ same as $T$, and remove the edge $e$ from
$T^{'}$. There is another edge $e^{'}$ on the same cycle of $G$ with $e$, and
$T^{'}$ does not have a cycle after add $e^{'}$ to $T^{'}$. Since
$w(e) \geq w(e^{'})$, so $w(T) \geq w(T^{'})$. Therefore, there is a minimum
spanning tree $T^{'}$ that does not include $e$.


\subsubsection {23.1-6}

Show that a graph has a unique minimum spanning tree if, for every cut of the
graph, there is a unique light edge crossing the cut. Show that the converse is
not true by giving a counterexample.

\textbf{Solution}

\textbf{1. Proof}

Let $T$ be the minimum spanning tree that is constructed by the unique light
edges crossing each cut. Assume that $T^{'}$ is another minimum spanning tree
which is different from $T$. We are going to show that the assume is
contradiction that $T^{'}$ can not be a minimum spanning tree.

Let $x$ be the vertex which has different edges in $T$ and $T^{'}$. Let edge
$(x, y_1)$ be the edge in $T$ but not in $T^{'}$. Let edge $(x, y_2)$ be the
edge in $T^{'}$ but not in $T$.

\textbf{Now we are going to show that $w(x, y_1) < w(x, y_2)$.} If we add the edge
$(x, y_2)$ into the $T$, there will be an cycle including the edge $(x, y_1)$
and $(x, y_2)$. Since there is a unique light edge for each cut, so
$w(x, y_1) != w(x, y_2)$. Assume $w(x, y_1) > w(x, y_2)$, so we will get a
better minimum spanning tree after replace $(x, y_1)$ with $(x, y_2)$. Since
$T$ is a minimum spanning tree, so there can not be a better minimum spanning
tree. So the assume that $w(x, y_1) > w(x, y_2)$ is contradiction. So
$w(x, y_1) < w(x, y_2)$.

If we add the edge $(x, y_1)$ into the $T^{'}$, there will an cycle including
the edge $(x, y_1)$ and $(x, y_2)$. Since $w(x, y_1) < w(x, y_2)$, we can get
a better minimum spanning tree if we replace $(x, y_2)$ with $(x, y_1)$. So the
$T^{'}$ is not a minimum spanning tree. Therefore, the assume that $T^{'}$ is
another minimum spanning tree which is different from $T$ is contradiction.

\textbf{2. Counterexample}

$G=(V, E)$ has three vertex: $A, B, C$ and two edges $(A, B), (A, C)$ which
$w(A, B) = w(A, C)$. There is a unique minimim spanning tree. However, the cut
of $\{A\}, \{B, C\}$ does not have a unique light spanning tree.

\subsubsection {23.1-7}

Argue that if all edge weights of a graph are positive, then any subset of edges
that connects all verticles and has minimum total weight must be a tree. Give an
example to show that the same conclusion does not follow if we allow some
weights to be nonpositive.

\textbf{Solution}

Firstly, we prove that the subset is a graph. Secondly, we prove that the subset
does not contain a cycle.

1) Since the subset of edges connect all verticles, the subset must be a graph.

2) Assume there is a cycle in the subset, since all the weights are positive,
if we remove one edge in the cycle, we will get a lesser total weight. However,
the subset has minimum total weight, so the assume is contradiction. So, there
is no cycle in the subset.

So the subset must be a tree.

\textbf{Counterexample}

$G = (V, E)$, $V = \{ A, B, C, D \}$,
$E = \{ (A, B), (B, C), (C, A), (A, D), (B, D), (C, D) \}$,
$w(A, B) = w(B, C), w(C, A) = -1$, $w(A, D) = 1, w(B, D) = 2, w(C, D) = 3$, the
minimum total weight is $w(A, B) + w(B, C) + w(C, A) + w(A, D) = -2$ but it has
a cycle.


{\color{red}the tree is a minimum spanning tree ??????}


\subsubsection {23.1-8}

Let $T$ be a minimum spanning tree of a graph $G$, and let $L$ be the sorted
list of the edge weights of $T$. Show that for any other minimum spanning tree
$T^{'}$ of $G$, the list $L$ is also the sorted list of edge weights of $T^{'}$.

\textbf{Solution-1}

We are going to replace different edges in $T^{'}$ with the same weight edges
in $T$. If finally $T^{'}$ is the same as $T$, then we are done.

1 \space \textbf{while} find a vertex $u$ in $T^{'}$ which only in two different
edges in $T^{'}$ and $T$ \\
2 \space \space \space Let $(u, x)$ in $T^{'}$ but not in $T$ \\
2 \space \space \space Let $(u, y)$ in $T$ but not in $T^{'}$ \\
2 \space \space \space There is a cut $(S, V - S)$ which $S$ includes $x, y$ and
excludes $u$. Since both the $T$ and $T^{'}$ are minimum spanning tree, so
$w(u, x) == w(u, y)$, so we can replace $(u, x)$ with $(u, y)$.\\
3 \space The final $T^{'}$ is the same as $T$, since each edge replaced in $T^{'}$
has the same weight as in $T$, so we are done.

\textbf{Solution-2}

\href{https://github.com/gzc/CLRS/}{Reference}

Let list $A = { a_1, a_2, ... , a_(i-1), a_i, a(i+1), ..., a_n }$ be the sorted
weights of $T$ in ascending order.

Let list $B = { b_1, b_2, ... , b_(i-1), b_i, b(i+1), ..., b_n }$ be the sorted
weights of $T$ in ascending order.

Assume there is a difference weight between $A$ and $B$ which is the ith, so
$a_i != b_i$. We are going to show that the assume is contradiction. We are
going to prove when the $a_i > b_i$, and it is also applied to $a_i < b_i$.

(1) If $b_i$ in the list $A$, then there is $a_j$ in the list $A$ and
$a_j == b_i$. Since $a_x == b_x$ when $x$ is from $1$ to $(i - 1)$, $j \geq i$,
so $b_i == a_j \geq a_i$, so $b_i \geq a_i$, so the assume is contradiction.

(2) If $b_i$ does not in the list $A$, there will a cycle in $T$ when we add
the edge of $b_i$. And $b_i$ is not less than any other weights in the cycle.
There is must a edge in the cycle which does not exist in $T^{'}$. Let the edge
be $a_x$ and $b_i \geq a_x$. Since $b_i \geq a_x \geq a_i$, the assume is
contradiction.

Therefore, the assume is contradiction, so there is not a difference weight
between $A$ and $B$.

\subsubsection{23.1-9}

Let $T$ be a minimum spanning tree of a  graph $G = (V, E)$, and let $V^{'}$ be
a subset of $V$. Let $T^{'}$ be the subgraph of $T$, and let $G^{'}$ be the
subgraph of $G$ induced by $V^{'}$. Show that if $T^{'}$ is connected, then
$T^{'}$ is a minimum spanning tree of $G^{'}$.

\textbf{Solution}

Firstly, we will prove that \textbf{when $T^{'}$ is connected, if we replace
$T^{'}$ with another tree $T^{'}_1$ that connects all the $V^{'}$ in $G^{'}$,
the $T$ of $G$ will be $T_1$, and $T_1$ is also a tree}. Assume that there is
cycle in $T_1$ after replace $T^{'}$ with $T^{'}_1$. Let $A, B, A^{'}, B^{'}$
in the cycle, and $A, B$ is in $V$ not in $V^{'}$, $A^{'}, B^{'}$ is in $V^{'}$,
and $A^{'}$ is the first vertex that $A$ connects $T^{'}$, $B^{'}$ is the first
vertex that $B$ connects $T^{'}$. Since $A, B$ is in the cycle, there must be
several edges that connect $A$ and $B$. Also, there must be several edges
connect $A^{'}$ and $B^{'}$. For all the trees in $G^{'}$ which connects all the
$V^{'}$, $A^{'}$ connects $B^{'}$, so does the $T^{'}$, so there is also a
cycle in $T$ which is a minimum spanning tree. So the assume is contradiction. 

Secondly, assume that there is a lesser weight tree than $T^{'}$ in $G^{'}$,
then replace $T^{'}$ with it, we will get a lesser weight tree in $G$.
Obviously, the assume is contradiction. So there is not a lesser weight tree in
$G^{'}$ than $T^{'}$. So $T^{'}$ is a minimum spanning tree of $G^{'}$.




\textbf{TODO}

TODO: Prim -> Kruskal

TODO: Kruskal -> Prim
