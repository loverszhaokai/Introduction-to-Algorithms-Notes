\section {Notes}

\begin{enumerate}[label=(\roman*)]
\item There maybe more than one MST in a forest.
\item The number of all the edges in the MST is equal to $\boldsymbol{V} - 1$.
\end{enumerate}

\section {Growing a minimum spanning tree}

\subsection {A}
A is a subset of some minimum spanning tree.

\subsection {Safe edge}
Safe edge is a edge that add to A and A is also a subset of some minimum
spanning tree.

\subsection {Generic-MST}

\textbf{GENERIC-MST}(G, $\omega$)\\
1\space A = $\varnothing$\\
2\space \textbf{while} A does not form a spanning tree\\
3\space\space\space\space find an edge($\mu$, $\upsilon$) that is
\textbf{safe edge} for A\\
4\space\space\space\space A = A $\cup$ {($\mu$, $\upsilon$)}\\
5\space return A\\
\\
\textbf{Initialization:} After line 1, the set A trivially satisfies the loop
invariant.\\
\textbf{Maintenance:} The loop in lines 2-4 maintains the invariant by adding
only safe edges.\\
\textbf{Termination:} All edges added to A are in a minimum spanning tree, and
so the set A returned in line 5 must be a minimum spanning tree.
