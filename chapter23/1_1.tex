\section {Notes}

\begin{enumerate}[label=(\roman*)]
\item There maybe more than one MST in a forest.
\item The number of all the edges in the MST is equal to $\boldsymbol{V} - 1$.
\end{enumerate}

\section {Growing a minimum spanning tree}

\subsection {Definition}

\subsubsection {A}
A is a subset of some minimum spanning tree.  

\subsubsection {Safe edge}
Safe edge is a edge that add to A and A is also a subset of some minimum
spanning tree.

\subsection {Generic-MST}

\textbf{GENERIC-MST}(G, $w$)\\
1\space $A$ = $\varnothing$\\
2\space \textbf{while} $A$ does not form a spanning tree\\
3\space\space\space\space find an edge $(u, v)$ that is
\textbf{safe edge} for $A$\\
4\space\space\space\space $A = A \cup \{(u, v)\}$\\
5\space return $A$\\
\\
\textbf{Initialization:} After line 1, the set A trivially satisfies the loop
invariant.\\
\textbf{Maintenance:} The loop in lines 2-4 maintains the invariant by adding
only safe edges.\\
\textbf{Termination:} All edges added to A are in a minimum spanning tree, and
so the set A returned in line 5 must be a minimum spanning tree.

\subsection {Theorem 1.}

Let $G=(V, E)$ be a connected, undirected graph with a real-valued weight
function $\omega$ defined on $E$. Let $A$ be a subset of $E$ that is inclued in
some minimum spanning tree for $G$, let $(S, V - S)$ be any cut of G that
respects A, and let $(u, v)$ be a light edge crossing $(S, V - S)$. Then, edge
$(u, v)$ is \textbf{safe} for A. {\color{red} Namely, $ A \cup {(u, v)}$ is also
included in some minimum spanning tree for $G$.}

\textbf{Proof} Let $T$ be a minimum spanning tree that includes $A$, and
{\color{red}assume that $T$ does not contain the light edge $(u, v)$}, since if
it does, the edge is obviously \textbf{safe} for A. We shall construct another
minimum spanning tree $T^{'}$ that includes $A \cup {(u, v)}$ by using
cut-and-paste technique, thereby showing that $(u, v)$ is a \textbf{safe} edge
for $A$.

The edge $(u, v)$ forms a {\color{red}cycle} with the edges on the simple path
$p$ from $u$ to $v$ in $T$. Since $u$ and $v$ are on opposite sides of the cut
$(S, V - S)$, at least one edge in $T$ lies on the simple path $p$ and also
crosses the cut. Let $(x, y)$ be any such edge. The edge $(x, y)$ is not in $A$,
because the cut respects $A$. Since $(x, y)$ is on the unique simple path from
$u$ to $v$ in $T$, removing $(x, y)$ breaks $T$ into two components. Adding
$(u, v)$ reconnects them to form a new spanning tree
$T^{'} = T - \{(x, y)\} \cup \{(u, v)\}$.

We next show that $T^{'}$ is a minimum spanning tree. Since $(u, v)$ is a light
edge crossing $(S, V - S)$ and $(x, y)$ also crosses this cut,
$w(u, v) \leq  w(x, y)$. Therefore,
$w(T^{'}) = w(T) - w(x, y) + w(u, v) \leq w(T)$.

When $w(T^{'}) == w(T)$, we know that $T^{'}$ is also a minimum spanning tree,
so the edge $(u, v)$ is \textbf{safe} for $A$.

When $w(T^{'}) < w(T)$, since we let $T$ be a minimum spanning tree and
\textbf{assume} that $T$ does not contain the light edge $(u, v)$. Therefore,
the \textbf{assume} is false, so $T$ must contain the light edge $(u, v)$, and
the edge $(u, v)$ is \textbf{safe} for $A$.

\subsection{Exercises}

\subsubsection {23.1-1}

Let $(u, v)$ be a minimum-weight edge in a connected graph $G$. Show that
$(u, v)$ belongs to some minimum spanning tree of $G$.

\textbf{Solution}

Let $E_u$ be all the edges that connected to the point $u$.

a. If there is only one edge connected to the point $u$, the edge belongs to
\textbf{all} the minimum spanning tree of $G$.

b. If there is more than one edge connected to the point $u$, we assume that
{\color{red} $(u, v)$ is not in any minimum spanning trees of $G$}. There must
be one edge $(u, x) { x != v}$ that is in some minimum spanning tree of $G$,
since $w(u, v) < w(u, x)$, therefore, the edge $(u, x)$ can not be in some
minimum spanning tree of $G$. So there is conflict and the assume is false. So,
the $(u, v)$ belongs to some minimum spanning tree of $G$.

\subsubsection {23.1-2}

Professor Sabatier conjectures the following converse of Theorem 1. in Minimum
Spanning Tree. Let $G = (V, E)$ be a connected, undirected graph with a
real-valued weight function $w$ defined on $E$. Let $A$ be a subset of $E$ that
is included in some minimum spanning tree for $G$, let $(S, V - S)$ be any cut
of $G$ that respects $A$,and let $(u, v)$ be a safe edge for $A$ crossing
$(S, V - S)$. Then, $(u, v)$ is a light edge for the cut. Show that the
professor’s conjecture is incorrect by giving a counterexample.

\textbf{Solution}

a. Here is a special case, the point $v$ of $(u, v)$ only has one edge, and
$w(u, v)$ is the largest, let $(x, y)$ be any other edge that crosses the cut,
obviously, $(u, v)$ is not a light edge for the cut.

b. Here is a generic case, assume that there is a light edge $(u^{'}, v^{'})$
crossing the cut and the edge has no common point with $(u, v)$, so
$w(u^{'}, v^{'}) < w(u, v)$. After combine $A$ with $(u^{'}, v^{'})$, there is
another $cut^{'}$ that crossing $(u, v)$, and it is a light edge for $cut^{'}$.
The previous case shows that $(u, v)$ is not a light edge for any cut but some
cut when $(u, v)$ is a safe edge for $A$.

\subsubsection {23.1-3}

Show that if an edge $(u, v)$ is contained in some minimum spanning tree, then
it is a light edge crossing some cut of the graph.

\textbf{Solution}

Let $T$ be the minimum spanning tree that contains the edge $(u, v)$, if we
remove the edge from $T$, and the other edges are $A$, obviously there is some
cut that crosses the edge $(u, v)$ which respects $A$. Then we are going to show
that the edge $(u, v)$ is a light edge crossing these cut.

If there is only one edge crossing the cut, obviously the edge $(u, v)$ is a
light edge crossing the cut.

If there is more than one edge crossing the cut, let $(x, y)$ be any edges
crossing the cut other than $(u, v)$. Assume that $w(x, y) < w(u, v)$, there
will another minimum spanning tree $T^{'}$ and
$w(T^{'}) = w(T) - \{(u, v)\} + \{(x, y)\} < w(T)$ which is impossible since the
$T$ is a minimum spanning tree. So the assume is contradiction and
$w(x, y) \geq w(u, v)$, so the edge $(u, v)$ is a light edge crossing some cut
of the graph.

\subsubsection {23.1-4}

Give a simple example of a connected graph such that the set of edges
\{$(u, v)$: there exists a cut $(S, V - S)$ such that $(u, v)$ is a light edge
crossing $(S, V - S)$\} does not form a minimum spanning tree.

\textbf{Solution}

There is a quadrangle: $V={A,B,C,D}, E={(A,B), (A,C), (B,C), (B,D), (C,D)}$
$, w(A,B)=w(A,C)=w(B,C)=1, w(B,D)=w(C,D)=2$. Obviously, $(A,B), (A,C) and (B,C)$
are lights edges crossing some cut. So the tree edges can join the set. And they
construct a circle, so the set can not form a minimum spanning tree.

I think if we add {\color{red} respect} to the set, then the set will form a
minimum spanning tee. Such as, \{$(u, v)$: there exists a cut $(S, V - S)$
{\color{red} which repects this set} such that $(u, v)$ is a light edge
crossing $(S, V - S)$\}, and the set will form a minimum spanning tree.






TODO: Prim -> Kruskal
TODO: Kruskal -> Prim
