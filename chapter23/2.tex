\section {The algorithms of Kruskal and Prim}

Both \textbf{Kruskal} and \textbf{Prim} use a specific rule to \textbf{determine
  a safe edge} in line 3 of GENERIC-MST.

In Kruskal's algorithm, the set \textbf{A} is a forest whose vertices are all
those of the given graph. The safe edge added to \textbf{A} is always a
\textbf{least-weight} edge in the graph that \textbf{connects two distinct
  components}.

In Prim's algorithm, the set \textbf{A} forms a single tree. The safe edge added
to \textbf{A} is always a \textbf{least-weight edge connecting the tree to a
  vertex not in the tree}.

\subsection {Kruskal}

Kruskal's algorithm finds a safe edge to add to the growing forest by finding,
of all the edges that connect any two trees in the forest, an edge $(u, v)$ of
least weight.

\textbf{MST-Kruskal}$(G, w)$\\
1\space\space $A$ = $\varnothing$\\
2\space\space \textbf{for} each vertex $v \in G.V$\\
3\space\space\space\space MAKE-SET$(v)$\\
4\space\space sort the edges of $G.E$ into nondecreasing order by weight $w$\\
5\space\space \textbf{for} each edge $(u, v) \in G.E$, taken in nondecreasing order by weight\\
6\space\space\space\space \textbf{if} FIND-SET($u$) $\neq$ FIND-SET($v$)\\
7\space\space\space\space\space\space\space $A = A \cup {(u, v)}$\\
8\space\space\space\space\space\space\space UNION$(u, v)$


{\color{red}TODO: time complexity after finish Chapter 21.}



\subsection {Prim}
